
\def\mytitle{MATRICES }
\def\myauthor{Akana Sai Kumar}
\def\contact{19pa1a0405@vishnu.edu.in}
\def\mymodule{Future Wireless Communication (FWC)}
\documentclass[10pt, a4paper]{article}
\usepackage[a4paper,outer=1.5cm,inner=1.5cm,top=1.75cm,bottom=1.5cm]{geometry}
\twocolumn
\usepackage{graphicx}
\graphicspath{{./images/}}
\usepackage[colorlinks,linkcolor={black},citecolor={blue!80!black},urlcolor={blue!80!black}]{hyperref}
\usepackage[parfill]{parskip}
\usepackage{lmodern}
\usepackage{tikz}
	\usepackage{physics}
%\documentclass[tikz, border=2mm]{standalone}
%\usepackage{karnaugh-map}
%\documentclass{article}
\usepackage{tabularx}
%\usepackage{circuitikz}
\usepackage{enumitem}
\usetikzlibrary{calc}
\usepackage{amsmath}
\usepackage{amssymb}
\renewcommand*\familydefault{\sfdefault}
\usepackage{watermark}
\usepackage{lipsum}
\usepackage{xcolor}
\usepackage{listings}
\usepackage{float}
\usepackage{titlesec}
\providecommand{\mtx}[1]{\mathbf{#1}}
\titlespacing{\subsection}{1pt}{\parskip}{3pt}
\titlespacing{\subsubsection}{0pt}{\parskip}{-\parskip}
\titlespacing{\paragraph}{0pt}{\parskip}{\parskip}
\newcommand{\figuremacro}[5]{
    \begin{figure}[#1]
        \centering
        \includegraphics[width=#5\columnwidth]{#2}
        \caption[#3]{\textbf{#3}#4}
        \label{fig:#2}
    \end{figure}
}

\newcommand{\myvec}[1]{\ensuremath{\begin{pmatrix}#1\end{pmatrix}}}
\let\vec\mathbf
\lstset{
frame=single, 
breaklines=true,
columns=fullflexible
}
\thiswatermark{\centering \put(10,-80){\includegraphics[scale=0.04]{logo}} }
\title{\mytitle}
\author{\myauthor\hspace{1em}\\\contact\\FWC22032\hspace{6.5em}IITH\hspace{0.5em}\mymodule\hspace{6em}Assignment 5}
\begin{document}
	\maketitle
	\tableofcontents
   \section{Problem}
Find the equation of the circle whose radius is 5 and which
touches the circle x2+y2-2x-4y-20=0 at the point (5, 5).
\section{Construction}
\includegraphics[scale=0.47]{fig.png}
\begin{center}
  Figure of construction
  	\end{center}
    The steps for constructing above figure are :
\begin{enumerate}
 \item Generate a circle1 of radius $r1$ with centre $\vec{B}$ 
 \item the circle2 whose radius  $r2$ is 5 and which touches circle1 at the point $\vec{P}$
 \item By using Section formula find the center of the cicle2 point $\vec{A}$
\item Find the equations of the circle2 by using standard equation of conics
\end{enumerate}

  	
  \section{Solution}

Circle equation : $x^2+y^2-2x-4y-20=0$\\
The standard equation of the conics is given as :
\begin{align}
\vec{x}^{\top}\vec{V}\vec{x}+2\vec{u}^{\top}\vec{x}+f=0
\end{align}
The given circle  can be expressed as conics with \\parameters
\begin{align}
	\vec{V} &= \vec{I}
	\end{align}
 \begin{align}
 \vec{u} = -\myvec{1 \\2}
 \end{align}
 \vspace{1mm}
 \begin{align}
 f = -20
	\end{align}
	Radius and Centre are
 \begin{align}
	r_1 &=\sqrt{{\vec{u}^{\top}\vec{u}}-f }
	\end{align}
    \begin{align}
 \vec{B}=-u
    \end{align}
    The input parameters for this construction are
\begin{center}
\begin{tabular}{|c|c|c|}
	\hline
	\textbf{Symbol}&\textbf{Value}&\textbf{Description}\\
	\hline
	$\vec{B}$ &\myvec{1\\2}& Centre of circle1\\
	\hline
    $r_2$ &{5}&radius of circle2\\
	\hline
 $\vec{P}$&\myvec{5\\5}&Point P\\
	\hline
\end{tabular}
\end{center}

\begin{align}
\vec{P} = \frac{\vec{A}(r_2) + \vec{B}(r_1)}{(r_1) +(r_2)}
\end{align}
\begin{align}
2\vec{P}-\vec{B} = \vec{A} 
\end{align}
\vspace{2mm}
to find "f2" from radius formula taking "A" as center we get 
\begin{equation}
f_2=\vec{u}^{\top}\vec{u} + r^2
\end{equation}
\vspace{2mm}
The standard equation of the conics is given as :
\begin{align}
	\vec{x}^{\top}\vec{V}_2\vec{x}+2\vec{u}_2^{\top}\vec{x}+f_2=0
\end{align}
\begin{align}
	\vec{V}_2 = \vec{I}
\end{align}
\begin{align}
\vec{A} = -u_2
\end{align}


\end{document}

