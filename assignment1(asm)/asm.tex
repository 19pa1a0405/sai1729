\def\mytitle{Boolean Expression to its simplest form using K-map}
\def\mykeywords{}
\def\myauthor{Akana Sai Kumar}
\def\contact{19pa1a0405@vishnu.edu.in}
\def\mymodule{Future Wireless Communication-(FWC22032)}
\documentclass[10pt, a4paper]{article}
\usepackage[a4paper,outer=1.5cm,inner=1.5cm,top=1.75cm,bottom=1.5cm]{geometry}
\twocolumn
\usepackage{graphicx}
\graphicspath{{./images/}}
\usepackage[colorlinks,linkcolor={black},citecolor={blue!80!black},urlcolor={blue!80!black}]{hyperref}
\usepackage[parfill]{parskip}
\usepackage{lmodern}
\renewcommand*\familydefault{\sfdefault}
\usepackage{watermark}
\usepackage{karnaugh-map}
\usepackage{lipsum}
\usepackage{xcolor}
\usepackage{listings}
\usepackage{float}
\usepackage{titlesec}
\usepackage{amsmath}
\usepackage{algorithm2e}

\titlespacing{\subsection}{0pt}{\parskip}{-3pt}
\titlespacing{\subsubsection}{0pt}{\parskip}{-\parskip}
\titlespacing{\paragraph}{0pt}{\parskip}{\parskip}
\newcommand{\figuremacro}[5]{
    \begin{figure}[#1]
        \centering
        \includegraphics[width=#5\columnwidth]{#2}
        \caption[#3]{\textbf{#3}#4}
        \label{fig:#2}
    \end{figure}
}

\lstset{
frame=single, 
breaklines=true,
columns=fullflexible
}
\thiswatermark{\centering \put(0,-90.0){\includegraphics[scale=0.05]{IITH.jpg}} }
\title{\mytitle}
\author{\myauthor\hspace{1em}\\\contact\\IITH\hspace{0.5em}-\hspace{0.5em}\mymodule}
\date{}
\hypersetup{pdfauthor=\myauthor,pdftitle=\mytitle,pdfkeywords=\mykeywords}
\sloppy
\begin{document}
  \maketitle
\tableofcontents

\section{Introduction}
K maps are used to  Simplify  Boolean Expressions the given Expression to solve 
F(X,Y,Z,W)=(0,1,4,5,6,7,8,9,11,15)

        



\section{karnaugh-map}
        \begin{karnaugh-map}[4][4][1][$ZW$][$XY$]
        \minterms{0,1,4,5,6,7,8,9,11,15}
        \maxterms{2,3,10,12,13,14}
        \implicant{0}{5}
        \implicantedge{0}{1}{8}{9}
        \implicant{4}{6}
        \implicant{15}{11}
        \end{karnaugh-map}

        F=X'Z'+Y'Z'+X'Y+XZW




\section{Components}



\begin{table}[htbp]
 \begin{center}
    \begin{tabular}{|l|c|c|c|c|c|c} \hline \textbf{Component}
  & \textbf{value} & \textbf{quantity} \\
 \hline
Resistor & 220 ohm & 1 \\ \hline
Arduino & UNO & 1 \\ \hline
LED &  & 1 \\ \hline
Bread board &  & 1 \\ \hline
Jumper wires & M-M & 10\\ \hline
\end{tabular}   
\end{center}
\caption{\label{table:dummytable} }
\end{table}


\section{Truth table for given expression}
\begin{table}[htbp]
 \begin{center}
    \begin{tabular}{|l|c|c|c|c|c|c|c|c} \hline \textbf{X}
  & \textbf{Y} & \textbf{Z} & \textbf{W}& \textbf{F} \\
 \hline
        0&0&0&0&1 \\
        \hline
        0&0&0&1&1 \\
        \hline
        0&0&1&0&0 \\
        \hline
        0&0&1&1&0 \\
        \hline
        0&1&0&0&1 \\
        \hline
        0&1&0&1&1 \\
        \hline
        0&1&1&0&1 \\
        \hline
        0&1&1&1&1 \\
        \hline
        1&0&0&0&1 \\
        \hline
        1&0&0&1&1 \\
        \hline
        1&0&1&0&0 \\
        \hline
        1&0&1&1&1 \\
        \hline
        1&1&0&0&0 \\
        \hline
        1&1&0&1&0 \\
        \hline
        1&1&1&0&0 \\
        \hline
        1&1&1&1&1 \\
        \hline
\end{tabular}   
\end{center}
\caption{\label{table:dummytable} }
\end{table}







\section{Connections and results}



Also make connections to arduino UNO ,led and inputs based on table3. 

\begin{table}[htbp]
 \begin{center}
    \begin{tabular}{|l|c|c|c|c|c|c|c|c} \hline \textbf{Arduino UNO}
  & \textbf{8} & \textbf{9} & \textbf{10}& \textbf{11}& \textbf{2}& \textbf{gnd} \\
 \hline
Input&X&Y&Z&W&&\\ \hline
led&&&&&+&- \\ \hline
\end{tabular}   
\end{center}
\caption{\label{table:dummytable} }
\end{table}


\begin{table}[htbp]
 \begin{center}
    \begin{tabular}{|l|c|c|c|c|c|c|c|c} \hline \textbf{Sample input}
  & \textbf{X} & \textbf{Y} & \textbf{Z}& \textbf{W}& \textbf{LED } \\
 \hline
1&0&0&0&0&ON\\ \hline
2&0&0&1&0&OFF \\ \hline
\end{tabular}   
\end{center}
\caption{\label{table:dummytable} }
\end{table}

\subsection{Code Link}
\vspace{5mm}
\begin{lstlisting}
https://github.com/19pa1a0405/sai1729/blob/main/assignment1(asm)/assembly.asm
\end{lstlisting}
\end{document}